\documentclass[12pt,a4paper,oneside,brazil]{abntex2}

% Pacotes que serão utilizados%
\usepackage{lmodern}
\usepackage[utf8]{inputenc}
\usepackage[brazil]{babel}
\usepackage[T1]{fontenc}
\usepackage{indentfirst}
\usepackage{graphicx}
\usepackage{microtype}
\usepackage{wrapfig}
\usepackage{amsmath}

% Informações do documento %
\title{Notas Econometria I}
\author{Thiago Oliveira Coelho}
\date{\today}



\begin{document}
\pagestyle{plain}
\pagenumbering{arabic}

\maketitle

\tableofcontents

\chapter{Introdução a séries de tempo}

Um processo estocástico (aleatório), ou uma série temporal, é uma sequência de variáveis aleatórias indexadas pelo tempo. Cada coleta de dados em um certo substrato de tempo é chamado de \emph{realização}. Estas séries podem ser estáticas ou dinâmicas.

\section{Estática}

\begin{equation}
 Y_t = \beta_0 + Z_t + u_t
\end{equation}

Uma série estática é aquela na qual o $Y_t$ é afetado por variáveis independentes somente no mesmo tempo que esta, ou seja, por $Z_t$.

\section{Não-Estática}

\begin{equation}
 Y_t = \beta_0 + Z_t + Z_{t-1} + u_t
\end{equation}
  
Nos modelos não estáticos, $Y_t$ responde a variáveis independentes de tempos anteriores, como por exemplo $Z_{t-1}$.

\subsection{Lag}
O \emph{lag} é a defasagem apresentada pelo modelo, o que significa que, por exemplo, um lag de ordem $5$, indica que a variável dependente é afetada pela variável independente nos últimos cinco períodos.

\subsection{Impacto}

\begin{equation}
\delta_1 = Y_t - Y_{t-1} 
\end{equation}

O $\delta$ é chamado de propensão de impacto, e demonstra a alteração em $Y$ decorrente de mudanças no tempo. No exemplo, é utilizado $\delta_1$, mas o substrato depende da distância entre as realizações. Por exemplo: $\delta_2 = Y_{t+2} - Y_{t}$. A soma de todos os $\delta$ resulta no que chamamos de \emph{Propensão de Longo Prazo (PLP)}:

\begin{equation}
    PLP = \delta_0 + \delta_1 + \delta_2 + ... + \delta_k
\end{equation}

\section{Tendência}

Se $Y$ possui uma tendência ao longo do tempo, se adiciona o valor de $t$ as variáveis independentes. Ao se identificar o efeito de \emph{sazonalidade}, de modo semelhante, se adiciona dessa vez uma \emph{dummy}, que indica quando a sazonalidade está presente. Se a variação é constante, admitimos que a tendência é exponencial, o que resulta num modelo da seguinte forma:

\begin{equation}
    Y_t = e^{\beta_{0t} + \beta_{1t} + ... + \beta_{nt} + u_t}
\end{equation}

Que ao ser linearizado se torna:

\begin{equation}
    \log{Y_t} = \beta_{0t} + \beta_{1t} + ... + \beta_{nt} + u_t
\end{equation}

Neste caso, podemos obter uma aproximação das taxas de crescimento com $\log{Y_t} - \log{Y_{t-1}}$.



\end{document}